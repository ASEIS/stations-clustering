
\section{Conclusions}

We present the result of a machine learning analysis using semi-supervised and supervised methods on a large dataset comparing synthetics and data based on multiple goodness-of-fit metrics used in ground motion validation with the goal of prioritize and reduce the number of metrics. As a result of the data processing analysis, we propose a simplified algorithm based on a decision tree which uses only two metrics as opposed to the initial eleven available in the dataset. In particular, the proposed algorithm uses a three disjunctive decisions based on comparisons between data and synthetics in terms of the total energy and acceleration response spectra. We also propose rules to allow for the new class-based validation criteria to combine results obtained from different components of motion, and demonstrate that the results obtained with the proposed algorithm using two metrics are comparable to those obtained with the score-based validation results used in other recent validation studies. Similarly, one could implement similar rules to combine results from a frequency-band analysis, using, for example, the mode or majority validation class. 

We recognize, however, that the proposed decision tree algorithm may not be a definitive one because of a potential bias on the fact that the dataset used here, although large enough from a statistical point of view, came from simulation done for a particular earthquake in a particular region. In a future follow-up study it would be ideal to refine the tree adding additional simulations from different earthquakes, regions, models, etcetera, in order to arrive to a sound and robust decision tree. The procedural steps laid out here, nonetheless, remains valid---and it is likely that the metrics of energy and response spectra will prevail as those among the most decisive ones.

This latter point is perhaps the most important conclusion of our study. Identifying the total energy and response spectra metrics as the decisive ones is a key contribution to validation of ground motion simulations. On the one hand, this contributes to clarifying a standing question in the area of validation, and on the other hand, it signals to simulators where the focus should be in the search for improvements in their models.

% OLD NAEEM VERSION
% ----------------- 

% In this study we use two machine learning algorithms to address the long-term question about choosing the appropriate metrics to evaluate the ground motion simulation process independent of application and components. Assuming hypothetical stations, we labeled the dataset through constraint \kmeans{} algorithm with subspace analysis. Using the labeled data we developed two decision trees algorithm to evaluate the ground motion simulation. We generate two models where the first model only uses two metrics to evaluate the stations and the second model which is more accurate and also more complicated than the first model uses 4 metrics. According to our analysis, Response Spectra is the most important metric in classifying the stations. After response spectra, Total Energy are mostly used. We studied different combination of components's classes and developed a model to predict a components independent class. We applied the models on 2008 Chino Hills earthquake simulation synthetics and data GOF scores and discussed the results. A comprehensive study is needed to address the significance of the results or variations to earthquake magnitude, velocity model, station distance from epicenter, earthquake depth, and so on. The proposed model could be used with high accuracy (about 90 \%) in determining the accuracy of simulation independent of application and components, uniformly among physics-based ground motion simulation developers.  