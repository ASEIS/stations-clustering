
\section{Conclusions}

We present the \change{results} of a machine learning analysis using semi-supervised and supervised methods on a large dataset comparing synthetics and data based on multiple goodness-of-fit metrics used in ground motion validation with the goal of \change{prioritizing} and \change{reducing} the number of metrics, and develop an application independent decision algorithm. As a result of the data processing analysis, we propose a simplified algorithm based on a decision tree which uses only two metrics as opposed to the initial eleven available in the dataset. In particular, the proposed algorithm uses three (disjunctive decision) steps based on the values obtained for the metrics of the total energy and acceleration response spectra. We also propose rules to allow for the new class-based validation criteria to combine results obtained from different components of motion, and demonstrate that the results obtained with the proposed algorithm using two metrics are comparable to those obtained with the score-based validation results used in other recent validation studies. \change{One} could implement similar rules to combine results from a frequency-band analysis, using, for example, the mode or majority validation class. 

We recognize, however, that the proposed decision tree algorithm may not be a definitive one because of a potential bias on the fact that the dataset used here, although large enough from a statistical point of view, came from simulations done for a particular earthquake, in a particular region\change{, and using a particular set of metrics}. In a future follow-up study, it would be ideal to refine the tree adding other simulation datasets (i.e., from different earthquakes, regions, models, \change{and using additional metrics)} in order to arrive to a sufficiently sound and robust decision tree. The procedural steps laid out here, nonetheless, \change{remain} valid\change{, and there is no reason why not to use it also in other contexts utilizing other metrics (e.g., structural responses metrics such as drift ratio) provided that they are properly normalized. In summary, we can say that for the case of the metrics used here,} we showed that the procedure and background information used for clustering and decision making is stable, and it is likely that\change{---despite the limitations just described---}the metrics of energy and response spectra (along with peak acceleration and velocity as suggested by the additional trees) will prevail as those among the most decisive ones.

\change{Identifying the total energy and response spectra metrics as decisive metrics in the comparisons between synthetics and observations is a key contribution to validation of ground motion simulations. This contributes to clarifying a standing question in the area of validation, and it provides an indication to simulators about where the focus in the search for improvements in their models.}
