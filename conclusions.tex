
\section{Conclusions}

In this study we use two machine learning algorithms to address the long-term question about choosing the appropriate metrics to evaluate the ground motion simulation process independent of application and components. Assuming hypothetical stations, we labeled the dataset through constraint \kmeans{} algorithm with subspace analysis. Using the labeled data we developed two decision trees algorithm to evaluate the ground motion simulation. We generate two models where the first model only uses two metrics to evaluate the stations and the second model which is more accurate and also more complicated than the first model uses 4 metrics. According to our analysis, Response Spectra is the most important metric in classifying the stations. After response spectra, Total Energy are mostly used. We studied different combination of components's classes and developed a model to predict a components independent class. We applied the models on 2008 Chino Hills earthquake simulation synthetics and data GOF scores and discussed the results. A comprehensive study is needed to address the significance of the results or variations to earthquake magnitude, velocity model, station distance from epicenter, earthquake depth, and so on. The proposed model could be used with high accuracy (about 90 \%) in determining the accuracy of simulation independent of application and components, uniformly among physics-based ground motion simulation developers.  