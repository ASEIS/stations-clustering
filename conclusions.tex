
\section{Conclusions}

In this study we use two machine learning algorithms to address the long time question about choosing the appropriate metrics to evaluate the ground motion simulation process. Assuming hypothetical stations, we labeled the dataset through constraint \kmeans{} algorithm with subspace analysis. Using the labeled data we develop a decision tree algorithm to evaluate the ground motion simulation. We generate two models where the first model only uses two metrics to evaluate the stations and the second model which is more accurate and also more complicated than the first model uses 5 metrics. According to our analysis, Response Spectra is the most important metric in classifying the stations. After response spectra, Total Energy are mostly used. We present the results for separate velocity models and components, however, a comprehensive study is needed to address the significance of the results or variations to earthquake magnitude, velocity model, component, station distance from epicenter, earthquake depth, and so on. The proposed model could be used with high accuracy in determining the accuracy of simulation uniformly among physics based ground motion simulation developers.  