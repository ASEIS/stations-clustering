
% \setlength{\paperwidth}{210mm}
% \setlength{\paperheight}{297mm}% fixed.

\documentclass{article}

\usepackage[review]{myreviewpckg}

\usepackage{textcomp}
\usepackage{simplemargins}

\clubpenalty=10000  % Orphan - First of paragraph left behind
\widowpenalty=10000 % Widow  - Last of paragraph sent ahead

\setallmargins{1in}

\begin{document}

% *****************************************************************
% *********************** Associate Editor ************************
% *****************************************************************

\begin{center}
	\bf
	\large
	Authors' Response to Associate Editor (Dr. Thomas Brocher)
\end{center}

\noindent
We thank Dr. Thomas Brocher for his comments and suggestions, which helped us improve the manuscript. To help the review process, in the annotated manuscript version of the paper, we have highlighted the most relevant changes that resulted from these comments in green font and the removed text in red font. In the following, we provide a response to each comment. The original comments are in italic black font, followed by our responses in regular blue font.
\vspace{2ex}
\newline

\introcomment{~}{%
I agree with the peer reviewers that this paper addresses a topical subject of great importance to BSSA readers. The science is in the process of transitioning from purely observational based analysis of ground motions to a mixture of observations and simulations of ground motions. The paper is a useful contribution to this discussion, is generally well written, and carefully describes the methodology used and its limitations. The peer reviewers have made some very helpful suggestions for improving the paper that I hope the authors will strongly consider.
}

\introcomment{~}{%
In particular, I agree with their comment that the limitations of the study, i.e., the choice of 11 metrics and earthquake simulations for only one earthquake, needs to be included in the abstract and the conclusions. In addition, I also agree that all the terms and methodology should be well defined for the BSSA readership, so that the paper can be understood by them without resort to frequent online searches.
}

\introcomment{~}{%
I found Figure 6 very helpful in giving me a very simple if heuristic understanding why C4, C5, C6, and C8 are the most diagnostic of the goodness of fit. It clearly shows that only these metrics are change significantly with the goodness of fit (P,F, G, E). Plotting these values as a function of P, F, G, and E together on the same figure might allow the reader to see this relationship more clearly and might help provide a heuristic understanding of how C4, C5, C6, and C8 are diagnostic whereas C11 is constant with P, F, G, and E, and thus not diagnostic. In fact, in my opinion, Figure 6 is a simple and easy way of determining which metrics are diagnostic that might even eliminate the need for the decision trees. In any case, more
discussion of this figure might help the authors make the case for the main conclusions.
}

\response{%
Thanks for the constructive suggestions. We agree with you and peer reviewers that the technical terms are not commonly used in seismology and strong motion earthquake engineering. The confusion happened because of integrating three different topics into a method to carry out this research. One technical term in one field is different in another field and that cause a confusion. For example cluster, class, and label all are referring to the same concept, however, it depends on the application. We added a paragraph in the \textbf{Data Analysis Method} section to elaborate this confusion.
\vspace{2ex}
\newline 
We also modified Figure 6 in order to easily track the variation of different metrics for final classes. We do agree with you that Figure 6 clearly gives a good understanding about the metrics, however, we add decision trees, because they can convert the qualitative judgment to quantitative conditional form to be able to exactly decide on the final class based on their numerical GOF scores.
\vspace{2ex}
\newline 
We added text to abstract and conclusion to give a heads up to readers about 11 metrics and one earthquake with 3 velocity model  simulations.
}


% ******************************************************************
% *************************** REVIEWER 1 ***************************
% ******************************************************************

\newpage
\begin{center}
	\bf
	\large
	Authors' Response to Reviewer 1
\end{center}

\noindent
We thank the reviewer for his/her comments and suggestions, which helped us improve the manuscript. To help the review process, in the annotated manuscript version of the paper, we have highlighted the most relevant changes that resulted from these comments in green font and the removed text in red font. In the following, we provide a response to each comment. The original comments are in italic black font, followed by our responses in regular blue font.
\vspace{2ex}
\newline

\introcomment{~}{%
This is an interesting article for the readers of the BSSA journal. The authors propose a new approach based on machine learning techniques to determine the importance of validation metrics for ground motion simulation. The field of ground motion simulation validation (GMSV) is relatively new and of interest to both seismologists who develop simulation models and engineers who would like to use these simulations in engineering applications. One of the most challenging questions to address is "which validation metric(s) is best to use?" as there are so many different metrics in this field and so many different opinions. It has been difficult to address this question because the answer varies depending on the application of the simulations (e.g., GMPE development, PSHA, structural analysis response of interest, etc). This paper is interesting because it provides a statistical approach to reduce the number of validation metrics and select a few representatives from a bigger set without compromising the final validation results. The authors have done a great job providing background on the topic. However, their application is very narrow, they only consider the 11 Anderson2004+ parameters (as they've noted there are other parameters) for one simulation method for one earthquake, and as a result their conclusions are only valid for this particular case. They have mentioned this limitation in the conclusions, but the abstract and introduction do not explain the limitations and the application-dependency of the proposed approach. Both Abstract and Introduction need modification to specify the application for which the results and conclusions are true (i.e., the application here is to narrow down the 11 Anderson2004+ parameters to select fewer parameters that give similar results as the original 11 for the simulations they chose in this paper). Currently, the Abstract and Introduction give the impression that the conclusions are true for any engineering application or simulation method. Furthermore, there should be a short discussion in the paper on how the users can apply the same method for different applications, for example for structural responses such as drift ratio or responses that are sensitive to high-frequency content of simulations (theoretically, this method should be applicable to any response of interest if they replace the GOF scores used in this study? If true this should be mentioned as it will be of interest to others).
}

\response{%
The reviewer pointed out a very important topic that we believe we did not discuss it well in the manuscript. The manuscript presents steps of moving from considerable amount of data with specialties knowledge (background knowledge) to a final product that gives a better understanding about the data. According the reviewer suggestion we added a paragraph  to discuss this important point. Also we modified abstract an introduction for acknowledging the limitations with the study. 
}

\end{document}

% \comment{1}{
% The terminologies used in the paper are not clear in many cases. For example, what are "features" "attributes" "dimensions" "trees, nodes and leafs" ``class''? It was very difficult to read the section on Data Analysis Method and Introduction since these terminologies were not clearly defined or described in terms of the variables used in this paper. I was confused what
% "data" and "results" were referring to: GOF score data? GOF metric data? GOF category? Something other than GOF? I suggest that the authors streamline the paper and the terminologies to be consistent and clear in the final version.
% }

% \response{%
% We agree with you that the technical terms are not commonly used in seismology and strong motion earthquake engineering. The confusion happened because of integrating three different topics into a method to carry out this research. One technical term in one field is different in another field and that cause a confusion. For example cluster, class, and label all are referring to the same concept, however, it depends on the application. We added a paragraph in the \textbf{Data Analysis Method} section to elaborate this confusion.\\
% }

% \comment{2}{%
% The terms "common practice" or "common standards" have been mentioned several times. Because GMSV is such a new field (and I don't think there is consensus, I for one don't agree
% with some statements but I agree that others may), I think these terms should be removed. Just say this is your opinion, or it is based on the references that you have listed.
% }

% \response{%
% Thanks for the comment. We fixed it. 
% }

% \comment{3}{%
% In abstract, "semi-supervised" and "supervised" learning techniques are mentioned, but in the paper these terms are not explained, what part of the learning techniques is semi-supervised
% and what does that mean?
% }

% \response{%
% We briefly explained these terms \textbf{Data Analysis Method} section. We also added clarification about semi-supervised and supervised learning into  \textbf{Introduction} section.
% }

% \comment{4}{%
% In introduction, the difference between "verification" and "validation" should be discussed.
% }

% \response{%
% A discussion about verification and validation is added. 
% }

% \comment{5}{%
% In abstract and introduction, what does a "direct application" mean? it's not clear. Similarly, in Validation Metrics section: "direct quantitative comparisons" and "indirect statistical analysis"
% }

% \response{%
% \color{red}{Ricardo, please address this comment.}
% }

% \comment{6}{%
% Page 4, line 53: add "relatively" before "well understood"
% }

% \response{%
% Added. Thanks.
% }

% \comment{7}{%
% The two questions posed in the introduction lines 59-61: I don't think the first one is really the question, all parameters should be considered if anyone has thought of them as important (i.e.,
% has proposed them). The second question is the main question and the one being addressed in this paper.
% }

% \response{%
% We modified the text for clarification and also to acknowledge each individual proposed metrics' importance. 
% }

% \comment{8}{%
% You are treating these metrics as "independent variables", but as you have also noted many of them are not. For example, Arias intensity and energy are correlated. Does this compromise your conclusions? If not explain why in the Results section.
% }

% \response{%
% This is a very interesting comment. You are totally right. There are fairly high correlation between Arias intensity and energy integral. However, decision tree algorithm is robust enough to realize this similarity and pick the best one in each decision making process. As we remove the peak ground acceleration and response spectra from the prediction model, the algorithm looks for other best parameters. In case of hundreds of metrics it would be a reasonable choice to do feature selection analysis and choose one metric out of several correlated metrics. However, in this study, we only deal with 11 metrics and the results show that it works fairly well. 
% } 


% \comment{9}{%
% Page 7, line 133: delete the last sentence
% }

% \response{%
% Deleted. Thanks.
% }

% \comment{10}{%
% Page 14, line 310: you say "... the latter can essentially be discarded..." I don't understand this statement, wouldn't this mean that both C1 and C7 are important to consider since they are
% independent of each other (one gives information that the other one doesn't)?
% }

% \response{%
% In that specific subspace (C1,C7), C1 does not help in clustering process. That is why we call it irrelevant feature, because one can define the clusters based on only defining boundaries on C7. C1 can be important, however, it does not add any positive or negative values in decision making process in this subspace analysis. 
% }

% \comment{11}{%
% Page 14 line 323: It's not clear how you "replicate data", provide more detail.
% }

% \response{%
% Thanks. More details provided. 
% }

% \comment{12}{%
% I don't think "confusion matrix" given in tables is explained in the paper!
% }

% \response{%
% In \textbf{Decision Trees} section we provided a definition of confusion matrix for the simplest case of two categories. The idea can be extended for more than two (in our case four) categories. 
% }

% \comment{13}{%
% Page 17, lines 391-394: this sentence should be re-worded, what is an "accurate conclusion"? You should be very specific that your conclusion is that instead of using all 11 validation metrics you can select a few in order to get the same categorization as all 11 metrics would have given you.}

% \response{%
% You are right. It could be misleading. We re-worded the sentence. 
% }

% \comment{14}{%
% Page 18, line 405: this is implying that fig 10 is similar to fig2! I personally cannot see this in the figures! They look very different. It might be better to regenerate fig 11a but with different color scales that only vary by four choices, then the comparison would be much easier.
% }

% \response{%
% That is correct. The statement convey part of the message. Although we expect to see similarities between the current results and previous studies, we keep in mind that this study is based on lack of consensus on using GOF metrics for validation purposes. Since the previous study is representing a uniform average of all metrics, we do not expect to match all stations. Therefore, we prefer to keep it as numerical. We modified the text accordingly. 
% }

% \comment{15}{%
% Figure 8: I don't know what you mean by y-label "decision attributes". Do you mean "the number of GOF metrics"? why there is no data point for 1 or 11?
% }

% \response{%
% That is right. It should be the number of decision attributes. We modified the figure accordingly. 
% }

% \comment{16}{%
% The conclusion basically says that response spectrum is the most important validation metric. This is not surprising to any engineer or modeler. The big question is what other parameters
% describe the differences between a simulation and a recorded motion if their response spectra matched? Or in other words, what are the other important validation parameters aside from
% response spectrum (many have argued that's duration, but no statistical proof). Can you use your method to answer this question? Can you condition your machine learning process on Sa
% being the same (for example use a subset of data with similar Sa or artificially make them equal), then decide what's the next important parameter? Would the result be the same as in your
% conclusions? Or would it be influenced by removal of Sa as a parameter (this is possible since there are correlations between Sa and other metrics).
% }

% \response{%
% That is a very good suggestion. We excluded C8(response spectra) and C5(peak acceleration) from analysis and reproduce similar trees with same complexities in terms of pruning process with previous four trees. As we expected, most important, metrics are still C4 and C6.  C11 also become somewhat effective in the most complicated tree. 
% }

% Typos:\\
% \\

% \comment{1}{%
% Page 2, line 8: change focused to focus
% }

% \response{%
% Done. Thanks.
% }

% \comment{2}{%
% Page 2, line 20: change classifiers to classifier
% }

% \response{%
% Fixed it. Thanks.
% }

% \comment{3}{%
% Page 13, line 185: change measure to measured
% }

% \response{%
% Changed. Thanks.
% }

% \comment{4}{%
% Page 16, line 354: the last C5 should be C6?
% }

% \response{%
% Fixed. Thanks.
% }

% % ******************************************************************
% % *************************** REVIEWER 2 ***************************
% % ******************************************************************

% \newpage

% \begin{center}
% 	\bf
% 	\large
% 	Authors' Response to Reviewer 2
% 	\end{center}

% \noindent
% We thank the reviewer for his/her comments and suggestions, which helped us improve the manuscript. To help the review process, in the annotated manuscript version of the paper, we have highlighted the most relevant changes that resulted from these comments in green font and the removed text in red font. In the following, we provide a response to each comment. The original comments are in italic black font, followed by our responses in regular blue font.
% \vspace{2ex}
% \newline

% \introcomment{~}{%
% Review of Khoshnevis and Taborda for BSSA
% }

% \introcomment{~}{%
% Earthquake ground motion simulations are usually validated against observational data or empirically-derived solutions. However, exactly which parameters to compare (e.g., peak
% ground velocity, duration, spectral content) in order to assess the goodness-of-fit, is not always clear. This paper uses machine learning techniques and eleven possible metrics for comparing (recorded and simulated) ground motions to determine the parameters that are most predictive of goodness-of-fit. This is an interesting paper, and will help future researchers
% prioritize which metrics to focus on when validating ground motions simulations. The paper is generally well written. As a strong ground motion seismologist (i.e., I assume I'm part of the audience you hope to reach), there is a lot of machine learning-related jargon that makes certain sections difficult to follow. It would also be useful to have some discussion regarding which parameters were found to be repetitive, incase future authors wish to look at more than the four recommended parameters.
% }

% \response{
% We agree with you that the technical terms are not commonly used in seismology and strong motion earthquake engineering. The confusion happened because of integrating three different topics into a method to carry out this research. One technical term in one field is different in another field and that cause a confusion. For example cluster, class, and label all are referring to the same concept, however, it depends on the application. We added a paragraph in the \textbf{Data Analysis Method} section to elaborate this confusion.\\
% We also studied the models without response spectra (C8) and without response spectra (C8) and peak ground acceleration (C5).\\ 
% }

% \comment{1}{%
% L. 8 "... could \textbf{focus} ..."
% }

% \response{
% Done. Thanks.
% }

% \comment{2}{%
% L. 11 "... the objective \textbf{of} ..."
% }

% \response{
% Added. Thanks.
% }

% \comment{3}{%
% L. 25-41 The frequent use of phrases such as "Among these methods,"; "Among the latter,"; "In this category,"; and "Within this group," make the text difficult to follow.
% }

% \response{
% \color{red}{Ricardo, please address this comment.}
% }

% \comment{4}{%
% L. 42-43 I would suggest moving the description of the Anderson (2004) method (i.e., L. 48+) to directly follow this sentence.}

% \response{
% Thanks for the suggestion. We moved the description of the Anderson (2004) method (i.e., L. 48+) to the mentioned location. It reads better now. 
% }

% \comment{5}{%
% L. 59-61/L. 391-394 It should be stated somewhere that this is really intended for \textbf{engineering purposes}.
% }

% \response{
% Thanks for pointing this out. Since Anderson developed metrics for engineering application, we assumed it is implicitly mentioned. However, we agree that it should be explicitly clarified. We modified the text accordingly. 
% }

% \comment{6}{%
% L. 61 This sentence should end with a "?"
% }

% \response{
% Fixed. Thanks.
% }

% \comment{7}{%
% L. 98-99 It is not clear how the equation for S is applied to values that are a function of period (e.g., the response or Fourier spectrum). Do you follow the guidelines provided by Anderson? If
% so, what frequency range/bands are you examining?
% }

% \response{
% In this study we only use broadband results. We added a clarification into the text. Thanks for the comment.  
% }

% \comment{8}{%
% L. 128/132/137/etc. "... the three component\textbf{s} ..."
% }

% \response{
% Fixed. Thanks.
% }

% \comment{9}{%
% L. 236 Where is the discussion on the consequences of overfitting? The only other place I see overfitting mentioned is L. 328.
% }

% \response{
% Added. Thanks.
% }

% \comment{10}{%
% L. 236 Remove "A matter we discuss later."
% }

% \response{
% Removed. Thanks.
% }

% \comment{11}{%
% L. 241-243 It is not helpful to list different algorithm names for making decision trees, without any information about them or how they differ.
% }

% \response{
% We added a brief description about them. Thanks.
% }

% \comment{12}{%
% L. 257-258 I don't understand this sentence. What is a continuous attribute?
% }

% \response{
% We meant to say numerical continues attribute. Thanks for the comment. Fixed. 
% }

% \comment{13}{%
% L. 259-264 This paragraph contains a lot of jargon.
% }

% \response{
% Thanks for the comment. Additional clarification added. 
% }

% \comment{14}{%
% L. 300 Isn't this also true for the C5-C7 combination?
% }

% \response{
% That is right. Added.
% }

% \comment{15}{%
% L. 328 What is "a strong pruning process"?
% }

% \response{
% We modified it to heavy pruning process and also added its explanation. 
% }

% \comment{16}{%
% L. 433 "We present the result\textbf{s} of..."
% }

% \response{
% Added. Thanks.
% }

% \comment{17}{%
% L. 435 "... goal of prioritiz\textbf{ing} and reduc\textbf{ing}..."
% }

% \response{
% Modified accordingly. Thanks.
% }

% \comment{18}{%
% L. 443 Remove "Similarly" or "similar" (redundant)
% }

% \response{
% Modified. Thanks.
% }

% \comment{19}{%
% L. 450 "... remain valid."
% }

% \response{
% Extra "s" deleted. Thanks.
% }

% \comment{20}{%
% L. 455 "... as the decisive \textbf{parameters} ..." ? (This short concluding paragraph would be much stronger without the use of unnecessary phrases, i.e., "This latter point", "One the one hand",
% "on the other hand."
% }

% \response{
% \color{red}{Ricardo, please address this comment.}
% }


% \end{document}
