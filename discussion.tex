
\section{Discussion}

According to the results Response Spectra (C8) is the most important metric in classifying the stations. Data percentage in Table.~\ref{tab:attribute_usage_1} represent the amount of data that the attribute is used in classifying them.  After C8, Arias Intensity Value is the most effective metric. With $C8 \leq 4.07~ \& ~ 4.07 < C8 \leq 6.1$ and $6.1 < C8~ \&~ 6.87 \leq C4$ and $6.87 \leq C4 ~ \&  6.1 \leq C8$ one can classify the simulation as poor, fair, good, and excellent, respectively, with a high confidence.  In this case minimum F1-score occurs for Class 3 and maximum F1-score occurs for Class 1. Pruning process reduce the effect of oversampling in cluster 4. In the second model, which is more accurate from the first model for all metrics, other than Response Spectra and Total Energy, 3 other metrics including: Peak Velocity, Peak Acceleration, and Cross Correlation, also determines the classes.  Fig.~\ref{fig:C8dist} represents the variation of Response Spectra score in different distances for different velocity models and components. As one can see it is fairly well distributed in different score variation in respect to response. There is differences in median value in different components. The median value improves from CVMH+without GTL layer to CVMS. 

\begin{figure}
    \centering
    \includegraphics
       % [width=\columnwidth]
        [width=\textwidth]
        {figures/pdf/Figure_15.pdf}
    \caption{Variation of C8 (Response Spectra) score with distance. Velocity models are represented in different shapes and components in different colors.  }
    \label{fig:C8dist}
\end{figure}

Analysis of statistical significance for relevance of results to earthquake, velocity model, frequency band, magnitude, distance of stations, and components needs comprehensive study beyond clustering and classification. However, we illustrate the effects without making strong decision whether they affect the results or not. More comprehensive study is undergoing with \citet{Taborda_2016_GJI} dataset. \\
Finally, we represent the application of these models on classifying stations on a geographical representation. Fig.~\ref{fig:M1_dtree_gof} represent the application of the first decision tree algorithm on the simulation of Chino Hills earthquake. In order to keep consistency in color code, and be able to compare the results with other published results \citep[e.g.,][]{Taborda_2013_BSSA, Taborda_2014_BSSA, Taborda_2016_GJI} we assign 3,5,7, and 9 to cluster 1(poor),2(fair),3(good), and 4(excellent), respectively. 

\begin{figure}
    \centering
    \includegraphics
       % [width=\columnwidth]
        [width=\textwidth]
        {figures/pdf/Figure_16.pdf}
    \caption{M1 GOF-score for 3 velocity models and 3 components for Chino Hills earthquake simulation (max =4Hz) }
    \label{fig:M1_dtree_gof}
\end{figure}


Subsequently, Fig.~\ref{fig:M2_dtree_gof} represent of application of M2 metric on data. 

\begin{figure}
    \centering
    \includegraphics
       % [width=\columnwidth]
        [width=\textwidth]
        {figures/pdf/Figure_17.pdf}
    \caption{M2 GOF-score for 3 velocity models and 3 components for Chino Hills earthquake simulation (max =4Hz) }
    \label{fig:M2_dtree_gof}
\end{figure}


These figures are important from different point of view. As we mentioned before and illustrated in different figures, the GOF results for different component could be different. This can happen for many reason, for example not accurate orientation of station or simply location of station where makes difference for vertical and horizontal incidents. The fact that which component we should use as a final decision is beyond the scope of this paper. However, using different data in clustering and generating classification models should not affect the results. We consider the GOF of two seismograms (data and synthetic) as an observation to study the relationship between metrics not components. We only distinguished data based on components for presentation purposes. In all steps we use all data. This argument is also correct for velocity model. A velocity model could result in different accuracy and GOF scores. Not surprisingly, the algorithm will classify them in different classes. The model predicts different results for the same pair of data and synthetic for different components and velocity models simply because they are different.

