% 
It has become common practice to validate ground motion simulation based on a variety of time- and frequency-domain metrics scaled to quantify the level of agreement between synthetics and data, or with respect to other reference solutions. There is, however, no agreement about the importance or weight that it ought to be given to each individual metric. This leads to their selection often being subjective, either based on intended applications or personal preferences. As a consequence, it is difficult for simulators to identify what modeling improvement are needed, which would be potentially easier if they could focused on a reduced number of alternative metrics. We present an analysis of ground motion validation metrics using semi-supervised and supervised learning techniques used to label and classify goodness-of-fit results, with the objective prioritizing and narrowing the choice of metrics available and commonly used in ground motion simulation validation. In particular, we study the relationships that exist between eleven different metrics. In the process carried out, these metrics are understood as part of a multi-dimensional space. We use a constrained \kmeans{} method, and conduct a subspace clustering analysis to address high-dimensional effects. This allows us to label a given validation dataset into four categories (poor, fair, good, and excellent) following common practice. We then develop a family of decision trees using a C5.0 algorithm, from which we select a few trees that help narrow the number of metrics leading to a validation prediction into the four categories. These decision trees can be understood as rapid predictors of the quality of a simulation, or as a data-informed classifiers that can help prioritize validation metrics. Our analysis indicates that among the eleven metrics considered, the acceleration response spectra and total energy of velocity are the most dominant ones, followed by the peak ground response in terms of velocity and acceleration.
