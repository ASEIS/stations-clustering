% 
It has become common practice to validate ground motion simulations based on a variety of time and frequency metrics scaled to quantify the level of agreement between synthetics and data, or with respect to reference solutions. There is, however, no agreement about the importance or weight that it ought to be given to each metric. This leads to their selection often being subjective, either based on intended applications or personal preferences. As a consequence, it is difficult for simulators to identify what modeling improvements are needed, which would be easier if they could \change{focus} on a reduced number of metrics. We present an analysis of ground motion validation metrics using semi-supervised and supervised learning techniques that help label and classify goodness-of-fit results, with the objective \change{of} prioritizing and narrowing the choice of metrics available and commonly used in ground motion simulation validation. \myrevision{We use a dataset of three physic-based regional scale ground motion simulations based on one earthquake and three velocity models with eleven goodness of fit metrics.} In particular, we study the relationships that exist between \myremove{eleven} \myrevision{these} different metrics. In the process carried out, these metrics are understood as part of a multi-dimensional space. We use a constrained \kmeans{} method, and conduct a subspace clustering analysis to address high-dimensional effects. This allows us to label a given validation dataset into four categories (poor, fair, good, excellent) following \myrevision{previous studies} \myremove{common practice}. We then develop a family of decision trees using a C5.0 algorithm, from which we select a few trees that help narrow the number of metrics leading to a validation prediction into the four categories. These decision trees can be understood as rapid predictors of the quality of a simulation, \change{or as data-informed classifiers} that can help prioritize validation metrics. Our analysis indicates that among the eleven metrics considered, the acceleration response spectra and total energy of velocity are the most dominant ones, followed by the peak ground response in terms of velocity and acceleration. \myrevision{Adding other simulation datasets (i.e., from different earthquakes, regions, models, metrics, etcetera) will improve the results in terms of developing a sufficiently sound and robust decision tree, however, the procedural steps laid out in this study, nonetheless, remain valid.}
